\documentclass[10pt]{article} %% What type of document you're writing.

 
\usepackage{graphicx}

 
\usepackage{hyperref}

 
\usepackage[dvipsnames]{xcolor}

\usepackage[utf8]{inputenc}

 
%%%%% Preamble

%% Packages to use

 
\usepackage{amsmath,amsfonts,amssymb} %% AMS mathematics macros

%% Title Information.

 
\documentclass[10pt]{article} %% What type of document you're writing.

 
\usepackage{graphicx}

 
\usepackage{hyperref}

 
\usepackage[dvipsnames]{xcolor} 

 
\usepackage{amsmath,amsfonts,amssymb} %% AMS mathematics macros

 

%% Title Information.


\title{VISION Y ALCANCE.\\ (Formula para calcular peso) \\ \\ \\ \\ \\ \\}


 
\author{Pedro de Jesús Hernández Rojas  \\ \\ \\ \\ \\ Universidad Veracruzana.  \\ \\ \\ \\ \\ Facultad de Negocios y Tecnologías, campus Ixtac.  \\ \\ \\ \\ \\ Pruebas de Software. \\ \\ \\ \\ \\ 501 Ingeniría de Software. \\ \\ \\ \\ \\ \\ \\}



 
%% \date{29 sep 2020} %% By default, LaTeX uses the current date

 
%%%%% The Document

 
\begin{document}
 
\maketitle


\section{Alcance y limitaciones.}
 
 \subsection*{Especificacion del alcance del software}

\textcolor{black}{Se busca la implementación del módulo de interfaz gráfica, el cual es importante porque será el medio de comunicación entre el usuario y el sistema.\\
Además de implementar el backend para que en conjunto con la interfaz gráfica se logré un funcionamiento correcto y agradable para los usuarios.
Tambien es importante la implementación del módulo de pruebas para asegurarse que el software resista/pase todas las pruebas necesarias para brindar un producto de calidad.}


\section{Limitaciones y exepciones.}

\textcolor{black}{Es relevante hacer mención de algunas de las implementaciones que estarán fuera de este proyecto:
}
\begin{itemize}
    \item Implementaciones revenibles:	
    \begin{itemize}
         \item Mala administracion del tiempo.
         \item Desconocer de las herramientas a utilizar.
    \end{itemize}
    \item Implementaciones durante el desarrollo del proyecto:
    \begin{itemize}
         \item Falsedad en mis habilidades para la elaboración del proyecto.
    \end{itemize}
    \item Implementaciones en carecimiento de apoyo:
    \begin{itemize}
         \item Falta de conocimiento de los interesados para definir claramente requerimientos.
         \item Compromiso para finalizar el proyecto
    \end{itemize}
\end{itemize}


\section{Visión del producto.} 
\newline
\newline

\subsection*{Visión}

 
\textcolor{black}{Para esta página web, brindar un resultado correcto, confiable y seguro, evitando a nuestros usuarios realizar el cálculo de forma manual, utlizando calculadora cientifica o lapiz y papel. De esta forma se optimizará el proceso de resolver la formula.} 


 
\subsection*{Características Importantes}

 
 \textcolor{black}{En esta página web se encontrarán características que son importantes mencionar antes de usarlas, para tener un mejor panorama del producto de software final y como interactuar con el de manera correcta.}
 \begin{itemize}
    \item Etiqueta con titulo.
    \item Etiqueta con la formula para calcular la energía.
    \item Etiqueta con el valor de la gravedad.
    \item Campo de texto para ingresar la masa.
    \item Botón para realizar el cálculo.
    \item Campo de texto para visualizar el resultado.
\end{itemize}

\subsection*{Suposiciones y dependencias}
\textcolor{black}{En los supuestos, se espera ofrecer:}
\begin{itemize}
    \item Proporcionar excelente nivel de calidad en su implementación.
    \item Se supone que el desarrollo no se pasará del tiempo establecido.
    \item Espero que sirva como herramienta de apoyo. 
\end{itemize}
\textcolor{black}{Dependencias:}
\begin{itemize}
    \item La aplicación web estará diseñada para su visualización en dispositivos que cuente con internet.
    \item Diseñada para ser una pagina web dinámica, en la cual el usuario se sienta comodo y confiado al relizar su consulta al calculo.
\end{itemize}
 

 
\end{document}